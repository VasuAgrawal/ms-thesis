\chapter{Conclusions and Future Work}

This work comprises the development of three separate payloads and a scalable and flexible software system running on each payload capable of providing rapid situational awareness in the form of Artifact Localizations for the DARPA Subterranean Challenge. Each of the three payloads (drone, Mk. 0, and Mk. 1) contains a variety of sensors and sensing modalities which are fused to provide a single, globally consistent list of artifacts to a human operator at the base station in a robust and timely manner. This simulates information being reported first responders and emergency personnel during a disaster scenario, where rapid and accurate information about the environment is critical to saving lives and mitigating damage.

The complete system was tested during a number of field experiments, culminating in a series of deployments during the Tunnel Circuit of the DARPA Subterranean Challenge. We reported 25 out of 40 artifacts correctly between the two portals (Safety Research, Experimental), and won first place out of the 11 teams at the event. We also won an award for reporting the most accurate artifact, a backpack artifact reported during the second deployment in the Safety Research portal, with an error of 0.18m. Using the data collected at the Tunnel Circuit, we demonstrate the advantage and necessity of the fusion of multiple sensors and sensing modalities, which results in more artifacts being found, and being found more quickly, than is possible with any single sensor.

Many avenues for improvement of the system exist. Hardware synchronization for timestamps between all sensors in the payloads would result in a more consistent registration of sensor data to state estimates. Explicit modeling of localization points on each artifact could help remove between 10 cm and 1 m of error for each artifact report. An improved dataset and object detection model would reduce the number of false positives detected, increasing precision. Adding a form of segmentation to object detection would remove points in the environment from the calculation of artifact locations. An environment-aware cell phone trilateration strategy would enable accurate cell phone localization in a multitude of environments and would remove the current requirement of proximity. The microphone array and LIDAR could be used to detect additional artifact categories, or provide additional evidence for existing ones. These changes, and many others, would improve the Artifact Localizations reported by the system and thus the quality of the situational awareness provided to future base station operators, including first responders and emergency personnel.